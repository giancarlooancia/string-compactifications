% $Header$

\documentclass{beamer}

% This file is a solution template for:

% - Talk at a conference/colloquium.
% - Talk length is about 20min.
% - Style is ornate.



% Copyright 2004 by Till Tantau <tantau@users.sourceforge.net>.
%
% In principle, this file can be redistributed and/or modified under
% the terms of the GNU Public License, version 2.
%
% However, this file is supposed to be a template to be modified
% for your own needs. For this reason, if you use this file as a
% template and not specifically distribute it as part of a another
% package/program, I grant the extra permission to freely copy and
% modify this file as you see fit and even to delete this copyright
% notice. 


\mode<presentation>
{
  \usetheme{Madrid}
  \usecolortheme{seahorse}
  \usecolortheme{rose}
  \usefonttheme[onlylarge]{structuresmallcapsserif}
  %\usefonttheme[onlysmall]{structurebold}
  \setbeamerfont{title}{shape=\itshape,family=\rmfamily}
  \setbeamercolor{title}{fg=red!80!black,bg=red!20!white}

  \setbeamercovered{transparent}
  % or whatever (possibly just delete it)
}


\usepackage[english]{babel}
% or whatever

\usepackage[latin1]{inputenc}
% or whatever

\usepackage[T1]{fontenc}
\usepackage{lmodern}
% Or whatever. Note that the encoding and the font should match. If T1
% does not look nice, try deleting the line with the fontenc.


\title % (optional, use only with long paper titles)
{Circle Compactifications and T-duality}

\subtitle
{Bosonic and Type II String}

\author % (optional, use only with lots of authors)
{Giancarlo Oancia}
% - Give the names in the same order as the appear in the paper.
% - Use the \inst{?} command only if the authors have different
%   affiliation.

\institute[University of Bologna] % (optional, but mostly needed)
{
  Department of Theoretical Physics\\
  University of Bologna}
% - Use the \inst command only if there are several affiliations.
% - Keep it simple, no one is interested in your street address.

\date[String Theory, 2025] % (optional, should be abbreviation of conference name)
{Presentation for String Theory Exam, 2025}
% - Either use conference name or its abbreviation.
% - Not really informative to the audience, more for people (including
%   yourself) who are reading the slides online

\subject{String Theory}
% This is only inserted into the PDF information catalog. Can be left
% out. 



% If you have a file called "university-logo-filename.xxx", where xxx
% is a graphic format that can be processed by latex or pdflatex,
% resp., then you can add a logo as follows:

 \pgfdeclareimage[height=0.5cm]{university-logo}{graphics/University_Crest.png}
 \logo{\pgfuseimage{university-logo}}



% Delete this, if you do not want the table of contents to pop up at
% the beginning of each subsection:
\AtBeginSubsection[]
{
  \begin{frame}<beamer>{Outline}
    \tableofcontents[currentsection,currentsubsection]
  \end{frame}
}


% If you wish to uncover everything in a step-wise fashion, uncomment
% the following command: 

%\beamerdefaultoverlayspecification{<+->}


\begin{document}

\begin{frame}
	\titlepage
\end{frame}

\begin{frame}{Outline}
	\tableofcontents
	% You might wish to add the option [pausesections]
\end{frame}


% Structuring a talk is a difficult task and the following structure
% may not be suitable. Here are some rules that apply for this
% solution: 

% - Exactly two or three sections (other than the summary).
% - At *most* three subsections per section.
% - Talk about 30s to 2min per frame. So there should be between about
%   15 and 30 frames, all told.

% - A conference audience is likely to know very little of what you
%   are going to talk about. So *simplify*!
% - In a 20min talk, getting the main ideas across is hard
%   enough. Leave out details, even if it means being less precise than
%   you think necessary.
% - If you omit details that are vital to the proof/implementation,
%   just say so once. Everybody will be happy with that.

\section{Field Theory Compactification}
\begin{frame}
  1
  
  \hyperlink{foo}{\beamerskipbutton{skip slide}}
  
  \end{frame}
  
  \begin{frame}
  2
  \end{frame}
  
  \begin{frame}[label=foo]
  3
  \end{frame}


\begin{frame}[t]
	\frametitle{There is no Largest Prime number}
	\framesubtitle{Subtitle}

	\begin{theorem}<1->
		There is no largest prime number
	\end{theorem}
	\begin{proof}<2->
		\begin{enumerate}
			\item<2-> Suppose $p$ were the largest prime number.
			\item<3-> Let $q$ be the product of the first $p$ numbers.
			\item<4-> Then $q + 1$ is not divisible by any of them.
			\item<2-> But $q + 1$ is greater than $1$, thus divisible by some prime
			      number not in the first $p$ numbers.\qedhere
		\end{enumerate}
	\end{proof}
	\uncover<5->{{The proof used \textit{reductio ad absurdum}.}}

\end{frame}

\begin{frame}
  \begin{description}
    \item[Prova] Ciao
    \item[Prova 2] Ciao  
  \end{description}
  
\end{frame}

\section{Bosonic String Compactification}

\subsection{Overview of the Bosonic String}

\begin{frame}{Make Titles Informative. Use Uppercase Letters.}{Subtitles are optional.}
	% - A title should summarize the slide in an understandable fashion
	%   for anyone how does not follow everything on the slide itself.

	\begin{itemize}
		\item
		      Use \texttt{itemize} a lot.
		\item
		      Use very short sentences or \emph{short} \alert{phrases}.
	\end{itemize}
	\begin{definition}
		A \alert{Prime number} is a number
	\end{definition}
	\pause
	\begin{example}
		\begin{itemize}
			\item this is prime

			\item this is \emph{not} prime

			\item this may be prime
		\end{itemize}
	\end{example}
\end{frame}

\begin{frame}
	\frametitle{What's Still To Do?}
	\begin{block}{Answered Questions}
		How many primes are there?
	\end{block}
	\begin{block}{Open Questions}
		Is every even number the sum of two primes?
	\end{block}
\end{frame}

\begin{frame}
	\frametitle{What's Still To Do?}
	\begin{itemize}
		\item Answered Questions
		      \begin{itemize}
			      \item How many primes are there?~\cite{Author1990}
		      \end{itemize}
		\item Open Questions
		      \begin{itemize}
			      \item Is every even number the sum of two primes?
		      \end{itemize}
	\end{itemize}
\end{frame}

\begin{frame}[fragile]
	\frametitle{An Algorithm For Finding Primes Numbers.}
	\begin{semiverbatim}
		\uncover<1->{\alert<0>{int main (void)}}
		\uncover<1->{\alert<0>{\{}}
		\uncover<1->{\alert<1>{ \alert<4>{std::}vector<bool> is_prime (100, true);}}
		\uncover<1->{\alert<1>{ for (int i = 2; i < 100; i++)}}
		\uncover<2->{\alert<2>{ if (is_prime[i])}}
		\uncover<2->{\alert<0>{ \{}}
		\uncover<3->{\alert<3>{ \alert<4>{std::}cout << i << " ";}}
		\uncover<3->{\alert<3>{ for (int j = i; j < 100;}}
		\uncover<3->{\alert<3>{ is_prime [j] = false, j+=i);}}
		\uncover<2->{\alert<0>{ \}}}
		\uncover<1->{\alert<0>{ return 0;}}
		\uncover<1->{\alert<0>{\}}}
	\end{semiverbatim}
	\visible<4->{Note the use of \alert{\texttt{std::}}.}
\end{frame}

\begin{frame}
	\frametitle{What's Still To Do?}
	\begin{columns}[t]
		\column{.5\textwidth}
		\begin{block}{Answered Questions}
			How many primes are there?
		\end{block}\pause
		\column{.5\textwidth}
		\begin{block}{Open Questions}
			Is every even number the sum of two primes?
			23
		\end{block}
	\end{columns}
\end{frame}


\section*{Summary}

\begin{frame}{Summary}

	% Keep the summary *very short*.
	\begin{itemize}
		\item
		      The \alert{first main message} of your talk in one or two lines.
		\item
		      The \alert{second main message} of your talk in one or two lines.
		\item
		      Perhaps a \alert{third message}, but not more than that.
	\end{itemize}

	% The following outlook is optional.
	\vskip0pt plus.5fill
	\begin{itemize}
		\item
		      Outlook
		      \begin{itemize}
			      \item
			            Something you haven't solved.
			      \item
			            Something else you haven't solved.
		      \end{itemize}
	\end{itemize}
\end{frame}



% All of the following is optional and typically not needed. 
\appendix
\section<presentation>*{\appendixname}
\subsection<presentation>*{For Further Reading}

\begin{frame}[allowframebreaks]
	\frametitle<presentation>{For Further Reading}

	\begin{thebibliography}{10}

		\beamertemplatebookbibitems
		% Start with overview books.

		\bibitem{Author1990}
		A.~Author.
		\newblock {\em Handbook of Everything}.
		\newblock Some Press, 1990.


		\beamertemplatearticlebibitems
		% Followed by interesting articles. Keep the list short. 

		\bibitem{Someone2000}
		S.~Someone.
		\newblock On this and that.
		\newblock {\em Journal of This and That}, 2(1):50--100,
		2000.
	\end{thebibliography}
\end{frame}

\end{document}


