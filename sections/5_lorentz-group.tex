To study the properties of the Lorentz and Poincaré groups, we start from their defining representations, keeping in mind what properties are representation-independent. In particular, we'll work in a 4-dimensional Minkowski spacetime with metric tensor 
\begin{equation}
    \eta = (\eta_{\mu\nu}) = \textup{diag}(+1,-1,-1,-1),
\end{equation}
and scalar product defined by
\begin{equation}
    x \cdot y \coloneq x^T \eta y = x^0y^0-\vec{x}\cdot\vec{y} = \eta_{\mu\nu} x^\mu y^\nu = x_\mu y^\mu .
\end{equation}

\color{red}L'ho scritto a caso come segnaposto, va scritta questa piccola introduzione. \color{blue}
We'll start from Lorentz group, in particular from its defining representation. We'll work out the algebra in the fundamental representation, taking infinitesimal transformation and computing the commutation relations. Then, we'll study the finite-dimensional representations, and see how to label them via SU(2) indices. We'll introduce spinor representation and explain why are they important in quantum mechanics. Finally, we'll introduce field representations, and explain how to construct a lorentz invariant action, and that a representation of the lorentz group is given by the nother charges under the symmetry. This will allow us to move forward and study the infinite dimensional representations of the Poincare group.
\color{black}


%%%%%%%%%%%%%%% LORENTZ GROUP %%%%%%%%%%%%%%%%%
\section{Lorentz Group}\label{sec:lorentz-group}
Lorentz transformations are those transformations $x \to x' = \Lambda x$ which leave the scalar product invariant, i.e.,
\begin{equation}
    (\Lambda x) \cdot (\Lambda y) = x \cdot y \implies x^T \Lambda^T \eta \Lambda y = x^T \eta y \implies \Lambda^T \eta \Lambda = \eta .
\end{equation}
Written in components, this condition becomes
\begin{equation}\label{eq:lorentz-transf-def-components}
    \eta_{\mu\nu} = \eta_{\alpha\beta} \tensor{\Lambda}{^\alpha_\mu} \tensor{\Lambda}{^\beta_\nu}.
\end{equation}

Since $\eta_{\mu\nu}$ is symmetric, this gives $10$ constraints. Further, since the Lorentz transformation is a $4 \times 4$ matrix, it depends on $16-10 = 6$ independent parameters, which will be interpreted later as three parameters for the boosts and three for rotations. 

For an infinitesimal transformation
\begin{equation}
    \tensor{\Lambda}{^\mu_\nu} \simeq \delta^\mu_\nu + \tensor{\omega}{^\mu_\nu} ,
\end{equation}
and using eq.~\eqref{eq:lorentz-transf-def-components}, we find
\begin{equation}\label{eq:parameters-lorentz}
    \omega_{\mu\nu} = -\omega_{\nu\mu} .
\end{equation}
\begin{mdframed}
\begin{innerproof}
    \begin{equation}
    \begin{split}
        \eta_{\mu\nu} &= \eta_{\alpha\beta} \tensor{\Lambda}{^\alpha_\mu} \tensor{\Lambda}{^\beta_\nu}
        \simeq \eta_{\alpha\beta} \left( \delta^\alpha_\mu + \tensor{\omega}{^\alpha_\mu} \right) \left(  \delta^\beta_\nu + \tensor{\omega}{^\beta_\nu} \right) 
        \\ &= \eta_{\alpha\beta} \delta^\alpha_\mu \delta^\beta_\nu + \eta_{\mu\beta} \tensor{\omega}{^\beta_\nu} + \eta_{\alpha\nu} \tensor{\omega}{^\alpha_\mu} + O(\omega^2)
        = \eta_{\mu\nu} + \omega_{\mu\nu} + \omega_{\nu\mu} + O(\omega^2) . \qedhere
    \end{split}
    \end{equation}
\end{innerproof}
\end{mdframed}

The transformations of a space with coordinates $\{y_1, \dots y_n, x_1, \dots, x_m\}$ which leave the quadratic form $( {y_1}^2 + \dots + {y_n}^2 ) - ( {x_1}^2 + \dots + {x_m}^2 )$ invariant define the orthogonal group $O(n,m)$. Thus, the \emph{Lorentz group} is $O(1,3)$.

The group axioms~\ref{def:group-axioms} are satisfied, and in particular there exists a unit element $\1$ and each $\Lambda$ has an inverse since its determinant is different from zero. Further, using that the determinant of a product is the product of the determinants, and that the transpose of a matrix has the same determinant as the matrix, one can verify
\begin{subequations}
\begin{gather}
    \Lambda^T \eta \Lambda = \eta \implies (\det \Lambda)^2 = 1 \implies \det \Lambda = \pm 1 , \\
    \eta_{\mu\nu} \tensor{\Lambda}{^\mu_0} \tensor{\Lambda}{^\nu_0} = (\tensor{\Lambda}{^0_0})^2 - \sum_k {(\tensor{\Lambda}{^k_0})}^2 = 1 \implies (\tensor{\Lambda}{^0_0})^2 \geq 1 .
\end{gather}
\end{subequations}

Depending on the signs of $\det \Lambda$ and $\tensor{\Lambda}{^0_0}$, the Lorentz group has four disconnected components. The subgroup with $\det \Lambda = 1$ and $\tensor{\Lambda}{^0_0} \geq 1$ is called \emph{proper orthochronous} Lorentz group, $SO(1,3)^+$. The other components can be constructed from a given $\Lambda \in SO(1,3)^+$ combined with space and/or time reflection.


%%%%%%%%%%%%%%% LORENTZ ALGEBRA %%%%%%%%%%%%%%%%%
\section{Lorentz Algebra}\label{sec:lorentz-algebra}
Let's now detach for a moment from any representation, considering the abstract group $SO(1,3)^+$ to study its algebra.

We've seen that the Lorentz group is characterised by six independent parameters, which can be collected into the antisymmetric matrix $\omega_{\mu\nu}$ (see eq.~\eqref{eq:parameters-lorentz}). It is then convenient to label the generators as $M^{\mu\nu} = -M^{\nu\mu}$, where each pair $(\mu,\nu)$ identifies a particular generator. Then, using the exponential map\footnote{As discussed in sec.~\ref{sec:lie-groups-algebras} after eq.~\eqref{eq:exp-map}, recall we assume the exponential map is surjective.\color{red}true?\color{black}}, any element $\Lambda \in SO(1,3)^+$ can be written as
\begin{equation}\label{eq:abstract-lorentz-group-element}
   \Lambda = e^{-\frac{i}{2} \omega_{\mu\nu} M^{\mu\nu}},
\end{equation}
(\color{red} is it a problem if we have minus sign here and before plus sign?\color{black}) with conventional choice of constants. Then, given a finite dimensional representation $(\rho, V)$ of dimension $n$, the group element~\eqref{eq:abstract-lorentz-group-element} is represented by the $n \times n$ matrix
\begin{equation}
    \Lambda_\rho = e^{-\frac{i}{2} \omega_{\mu\nu} M^{\mu\nu}_\rho},
\end{equation}
which acts on $V$, where $M^{\mu\nu}_\rho$ are the Lorentz group generators in the representation $\rho$. Further, the elements of $V$ transform, under a Lorentz transformation, as
\begin{equation}
    \Phi^i \to \tensor{\left[ e^{-\frac{i}{2} \omega_{\mu\nu} M^{\mu\nu}_\rho} \right]}{^i_j}  \, \Phi^j .
\end{equation}

For an infinitesimal Lorentz transformation, with infinitesimal parameters $\omega_{\mu\nu}$, the variation of $\Phi^i$ is
\begin{equation}\label{eq:action-lorentz-transf}
    \delta \Phi^i = -\frac{i}{2} \omega_{\mu\nu} \tensor{\left(  M^{\mu\nu}_\rho \right)}{^i_j} \, \Phi^j, 
\end{equation}
where in $\tensor{\left(  M^{\mu\nu}_\rho \right)}{^i_j}$, the indices $\mu,\nu$ identify the generator\footnote{Beware the index $\rho$! It isn't a Lorentz index, but it stands for “representation”.}, while the indices $i,j$ are the matrix indices of the representation which has been considered. We can then classify all physical quantities according to their transformation properties under the Lorentz group.

For clarity, we anticipate the Lie algebra generators' commutation relations, which are representation independent; we'll further compute them explicitly for the four-vector representation. The Lie algebra $\mathfrak{so}(1,3)$ is characterised by
\begin{equation}\label{eq:lorentz-algebra-relations}
    \comm{M^{\mu\nu}}{M^{\rho\sigma}} = i \left( \eta^{\nu\rho} M^{\mu\sigma} - \eta^{\mu\rho} M^{\nu\sigma} - \eta^{\sigma\mu} M^{\rho\nu} + \eta^{\sigma\nu} M^{\rho\mu} \right).
\end{equation}

It is convenient to rearrange the generators into two spatial vectors\footnote{To see how to invert expressions with Levi-Civita symbol, look at appendix~\ref{app:levi-civita}.},
\begin{equation}\label{eq:redef-lorentz-gen}
    J^i = \frac{1}{2} \epsilon^{ijk} M^{jk} \iff M^{ij} = \epsilon^{ijk} J^k, \quad K^i = M^{i0} .
\end{equation}

In terms of $\vec{J}$ and $\vec{K}$, the Lie algebra of the Lorentz group reads (\color{red}To be computed!\color{black})
\begin{subequations}
\begin{align}
    \comm{J^i}{J^j} &= i \epsilon^{ijk} J^k \label{eq:lorentz-algebra-rotation} \\ 
    \comm{J^i}{K^j} &= i \epsilon^{ijk} K^k \label{eq:lorentz-algebra-boost-vector} \\ 
    \comm{K^i}{K^j} &= -i \epsilon^{ijk} J^k .
\end{align}
\end{subequations}

Equation~\eqref{eq:lorentz-algebra-rotation} is the Lie algebra of $SU(2)$ and this shows that $\vec{J}$ can be interpreted as the angular momentum. Instead, eq.~\eqref{eq:lorentz-algebra-boost-vector} expresses the fact that $\vec{K}$ is a spatial vector, since it transforms accordingly under a rotation. 

Defining, further
\begin{equation}\label{eq:lorentz-redef-param}
    \theta^i = \frac{1}{2} \epsilon^{ijk} \omega^{jk} \iff \omega^{ij} = \epsilon^{ijk} \theta^k, \quad \eta^i = \omega^{i0} ,
\end{equation}
a generic Lorentz transformation can be written as
\begin{equation}
    \Lambda = e^{-i \vec{\theta} \cdot \vec{J} + i \vec{\eta} \cdot \vec{K}} .
\end{equation}
\begin{mdframed}
\begin{innerproof}
    \begin{equation*}
    \begin{split}
        \frac{1}{2} \omega_{\mu\nu} M^{\mu\nu} &= \frac{1}{2} \Bigl( \sum_{i=1}^3 \left( \omega_{i0} M^{i0} + \omega_{0i} M^{0i} \right) + \omega_{12} M^{12} + \omega_{21} M^{21} \\
        &\qquad \quad \; \; + \omega_{13} M^{13} + \omega_{31} M^{31} + \omega_{23} M^{23} + \omega_{32} M^{32} \Bigr) \\
        &= \omega^{12} M^{12} + \omega^{13} M^{13} + \omega^{23} M^{23} - \sum_{i=1}^3 \omega^{i0} M^{i0} \\
        &= \vec{\vec{\theta} \cdot \vec{J} - \vec{\eta} \cdot \vec{K}},
    \end{split}
    \end{equation*}
    where we've used at the same time the antisymmetries $\omega_{\mu\nu} = - \omega_{\nu\mu}$ and $M^{\mu\nu}=- M^{\nu\mu}$. Moreover, according to~\eqref{eq:lorentz-redef-param}, $\omega_{i0} = -\omega^{i0}=-\eta^i$, $\omega_{12} = \omega^{12} = \theta^3$, $\omega_{13} = \omega^{13} = - \theta^2$ and $\omega_{23} = \omega^{23} = \theta^1$, and according to~\eqref{eq:redef-lorentz-gen}, $M_{i0} = -M^{i0}=- K^i$, $M_{12} = M^{12} = J^3$, $M_{13} = M^{13} = - J^2$ and $M_{23} = M^{23} = J^1$.

    Substituting into eq.~\eqref{eq:abstract-lorentz-group-element} leads to the result.
\end{innerproof}
\end{mdframed}

Let's now classify the representations of the Lorentz group.


%%%%%%%%%%%%%%%  TRIVIAL REPRESENTATION %%%%%%%%%%%%%%%%%
\subsection{Trivial Representation}
This is the one acting on a \emph{scalar} $\phi$, i.e., a quantity which is invariant under a Lorentz transformation, like the rest mass of a particle. By means of~\eqref{eq:action-lorentz-transf}, we have
\begin{equation}\label{eq:lorentz-transf-scalar}
    \delta \phi = 0,
\end{equation}
so that the generators, which are $1 \times 1$ matrices, identically vanish,
\begin{equation}
    \M^{\mu\nu} \equiv 0.
\end{equation}

The representation is called \emph{trivial}, since the algebra commutation relations~\eqref{eq:structure-contants} are trivially satisfied.

%%%%%%%%%%%%%%%  VECTOR REPRESENTATION %%%%%%%%%%%%%%%%%
\subsection{Vector Representation}\label{sec:vector-representation}
This is the \emph{defining representation} and acts on a \emph{four-vector} $V^\mu$, which has transformation law
\begin{equation}\label{eq:lorentz-transf-four-vector}
    V^\mu \to \tensor{\Lambda}{^\mu_\nu} V^\nu,
\end{equation}
where $\Lambda$ satisfy the condition~\eqref{eq:lorentz-transf-def-components}, i.e.,
\begin{equation*}
    \eta_{\mu\nu} = \eta_{\alpha\beta} \tensor{\Lambda}{^\alpha_\mu} \tensor{\Lambda}{^\beta_\nu}.
\end{equation*}

One could generically call the previous vector a \emph{contravariant} one and define a \emph{covariant} four-vector $V_\mu$ as one which transforms as $V_\mu \to \tensor{\Lambda}{_\mu^\nu} V_\nu$, with $\tensor{\Lambda}{_\mu^\nu} = \eta_{\mu\alpha} \eta^{\nu\beta} \tensor{\Lambda}{^\alpha_\beta} = \tensor{(\Lambda^{-1})}{^\mu_\nu}$. However, this distinction is not useful, since the two are related via index raising/lowering by the metric tensor, $V_\mu = \eta_{\mu\nu} V^\nu$. An example of four-vector is given by the four-momentum.

To find the explicit representation acting on $V^\mu$, let's first notice that the indices $i,j$ of eq.~\eqref{eq:action-lorentz-transf} are now Lorentz indices themselves, so that the representation matrix would read $\tensor{(M^{\mu\nu})}{^\alpha_\beta}$. Here, we drop any index referring to the representation. If not explicitly indicated, the context should make evident if we're talking about the abstract generator or a representation. 

Then, a four-vector transforms as
\begin{equation}\label{eq:variation-lorentz-transf-four-vector}
   \delta V^\alpha = - \frac{i}{2} \omega_{\mu\nu} \tensor{(M^{\mu\nu})}{^\alpha_\beta} V^\beta ,
\end{equation}
where
\begin{equation}\label{eq:lorentz-transf-matrix-vector-rep}
    \tensor{(M^{\mu\nu})}{^\alpha_\beta} = i \left( \eta^{\mu\alpha} \delta^\nu_\beta - \eta^{\nu\alpha} \delta^\mu_\beta \right) .
\end{equation}

\begin{mdframed}
\begin{innerproof}
    Considering the infinitesimal version of eq.~\eqref{eq:lorentz-transf-four-vector}, we get
    \begin{equation*}
        V^\alpha \to \tensor{\Lambda}{^\alpha_\beta} V^\beta \simeq (\delta^\alpha_\beta + \tensor{\omega}{^\alpha_\beta}) V^\beta,
    \end{equation*}
    which gives an infinitesimal variation
    \begin{equation*}
    \begin{split}
        \delta V^\alpha &= \tensor{\omega}{^\alpha_\beta} V^\beta = \frac{1}{2} \omega_{\mu\beta} \left( \eta^{\mu\alpha} V^\beta - \eta^{\beta\alpha} V^\mu \right) = \frac{1}{2} \omega_{\mu\nu} \delta^\nu_\beta \left( \eta^{\mu\alpha} V^\beta - \eta^{\beta\alpha} V^\mu \right) \\
        &= \frac{1}{2} \omega_{\mu\nu} \left( \delta^\nu_\beta \eta^{\mu\alpha} - \delta^\mu_\beta \eta^{\nu\alpha} \right) V^\beta \overset{!}{=} - \frac{i}{2} \omega_{\mu\nu} \tensor{(M^{\mu\nu})}{^\alpha_\beta} V^\beta ,
    \end{split}
    \end{equation*}
    where the second addend was added to ensure the piece within parenthesis is antisymmetric with respect to $\mu$--$\beta$, as should be by eq.~\eqref{eq:parameters-lorentz}. Then, in the second to last passage we relabelled a couple of dummy indices, to compare the expression with eq.~\eqref{eq:variation-lorentz-transf-four-vector}. We obtain the matrix~\eqref{eq:lorentz-transf-matrix-vector-rep}.
\end{innerproof}
\end{mdframed}

It's now easy to compute the commutators of~\eqref{eq:lorentz-transf-matrix-vector-rep} to find Lorentz algebra~\eqref{eq:lorentz-algebra-relations}. As already stated, the structure constants don't depend on the representation, so what we find here are the abstract commutation relations.
\begin{mdframed}
\begin{innerproof}
    Using eq.~\eqref{eq:lorentz-transf-matrix-vector-rep}, we compute
    \begin{equation*}
    \begin{split}
        \tensor{\comm{M^{\mu\nu}}{M^{\rho\sigma}}}{^\alpha_\beta} &= \tensor{(M^{\mu\nu})}{^\alpha_\gamma} \tensor{(M^{\rho\sigma})}{^\gamma_\beta} - \tensor{(M^{\rho\sigma})}{^\alpha_\gamma} \tensor{(M^{\mu\nu})}{^\gamma_\beta}  \\
        &= - (\eta^{\mu\alpha} \delta^\nu_\gamma - \eta^{\nu\alpha} \delta^\mu_\gamma) (\eta^{\rho\gamma} \delta^\sigma_\beta - \eta^{\sigma\gamma} \delta^\rho_\beta) + (\rho \leftrightarrow \mu, \; \sigma \leftrightarrow \nu) \\
        & = -\eta^{\mu\alpha} \eta^{\rho\nu} \delta^\sigma_\beta + \eta^{\mu\alpha} \eta^{\sigma\nu} \delta^\rho_\beta + \eta^{\nu\alpha} \eta^{\rho\mu} \delta^\sigma_\beta - \eta^{\nu\alpha} \eta^{\sigma\mu} \delta^\rho_\beta \\
        &\quad + \eta^{\rho\alpha} \eta^{\mu\sigma} \delta^\nu_\beta - \eta^{\rho\alpha} \eta^{\nu\sigma} \delta^\mu_\beta - \eta^{\sigma\alpha} \eta^{\mu\rho} \delta^\nu_\beta + \eta^{\sigma\alpha} \eta^{\nu\rho} \delta^\mu_\beta \\
        &= - \eta^{\nu\rho} ( \eta^{\mu\alpha} \delta^\sigma_\beta - \eta^{\sigma\alpha} \delta^\mu_\beta ) + \eta^{\mu\rho} ( \eta^{\nu\alpha} \delta^\sigma_\beta - \eta^{\sigma\alpha} \delta^\nu_\beta ) \\
        &\quad + \eta^{\sigma\mu} ( \eta^{\rho\alpha} \delta^\nu_\beta - \eta^{\nu\alpha} \delta^\rho_\beta) - \eta^{\sigma\nu} (\eta^{\rho\alpha} \delta^\mu_\beta - \eta^{\mu\alpha} \delta^\rho_\beta) \\
        &= i  \eta^{\nu\rho} \tensor{(M^{\mu\sigma})}{^\alpha_\beta}  - i \eta^{\mu\rho} \tensor{(M^{\nu\sigma})}{^\alpha_\beta} - i \eta^{\sigma\mu}  \tensor{(M^{\rho\nu})}{^\alpha_\beta} +  i \eta^{\sigma\nu} \tensor{(M^{\rho\mu})}{^\alpha_\beta} . \qedhere
    \end{split}
    \end{equation*}
\end{innerproof}
\end{mdframed}

As previously noticed, the Lie algebra $\mathfrak{so}(1,3)$ contains the Lie algebra $\mathfrak{so}(3)$. This is to be expected. Indeed, for Lorentz transformation of the form $\Lambda = \left(\begin{smallmatrix} 1 & 0 \\ 0 & R \end{smallmatrix} \right) \in SO(1,3)^+$, with $R$ a $3 \times 3$ matrix, the defining property~\eqref{eq:lorentz-transf-def-components} reduced to $R^TR = \1$, which is the defining property of $SO(3)$. Then, $SO(3)$ is a subgroup of $SO(1,3)^+$ and consequently the Lie algebras must reflect this property. From the explicit representation of the generators~\eqref{eq:lorentz-transf-matrix-vector-rep}, we can see that $M_{ij}$ are block diagonal matrices of the form $M \sim \left( \begin{smallmatrix} 1 & 0 \\ 0 & N \end{smallmatrix} \right) $, with $N$ a $3 \times 3$ matrix, so the exponentiation conserves this form and produces the expected rotation matrices inside $SO(1,3)^+$.


%%%%%%%%%%%%%%%  TENSOR REPRESENTATION %%%%%%%%%%%%%%%%%
\subsection{Tensor Representation}
Let's briefly recall what a tensor is. If $V$ is an $n$ dimensional real vector space and $V^*$ its dual space, the space of \emph{type $(p,q)$-tensors} is defined as
\begin{equation}
    T^{(p,q)}(V) = 
    \underbrace{V \otimes \dots \otimes V}_\text{$p$ times}
    \otimes
    \underbrace{V^* \otimes \dots \otimes V^*}_\text{$q$ times} .
\end{equation}

As we know from multilinear algebra, if $\{e_i\}$ is any basis for $V$ and $\{\epsilon^j\}$ the corresponding dual basis for $V^*$, a basis for $T^{(p,q)}(V)$ is given by
\begin{equation}
    \{ e_{i_1} \otimes \dots \otimes e_{i_p} \otimes \epsilon^{j_1} \otimes \dots \otimes \epsilon^{j_q} : 1 \leq i_1, \dots, i_p, j_1, \dots, j_q \leq n \},
\end{equation}
and the tensor can be written as
\begin{equation}
    T = \tensor{T}{^{i_1}^\dots^{i_p}_{j_1}_\dots_{j_q}} e_{i_1} \otimes \dots \otimes e_{i_p} \otimes \epsilon^{j_1} \otimes \dots \otimes \epsilon^{j_q}.
\end{equation}

We can uniquely identify a tensor via its coordinates, so we'll forget about the basis elements from now on. Further, we'll consider only \emph{contravariant} tensors, i.e., tensors with upper indices, since we can easily lower indices via the metric tensor and transform a \emph{covariant} index with the inverse of the metric, as already discussed for vectors, which are rank $1$ tensors.

Then, a Lorentz tensor of \emph{rank} $n$ is defined by the transformation law
\begin{equation}\label{eq:lorentz-transf-tensor}
    T^{\mu\nu\dots\tau} \to {T'}^{\mu\nu\dots\tau} = \underbrace{\tensor{\Lambda}{^\mu_\alpha} \tensor{\Lambda}{^\nu_\beta} \dots \tensor{\Lambda}{^\tau_\lambda}}_\text{$n$ times} T^{\alpha\beta\dots\lambda},
\end{equation}
so we can always construct the representation matrices $\tensor{\Lambda}{^\mu_\alpha} \tensor{\Lambda}{^\nu_\beta} \dots$ of the Lorentz transformation as the outer product $\vec{4} \otimes \vec{4} \otimes \dots$ of the $4$ dimensional defining representation $\Lambda$. However, these representations are not irreducible. To see why, let's study the simplest case of a $4 \times 4$ tensor $T^{\mu\nu}$, which has $16$ components.

Its trace, its antisymmetric component, and its symmetric and traceless part,
\begin{equation}
   S = \tensor{T}{^\alpha_\alpha} , \quad A^{\mu\nu} = \frac{1}{2} (T^{\mu\nu} - T^{\nu\mu}), \quad S^{\mu\nu} = \frac{1}{2} (T^{\mu\nu} + T^{\nu\mu}) - \frac{1}{4} \eta^{\mu\nu} S, 
\end{equation}
don't mix under Lorentz transformations, since a (anti-) symmetric tensor is still (anti-) symmetric after the transformation, and the trace is a Lorentz scalar.
\begin{mdframed}
\begin{innerproof}
    For the trace, using the property~\eqref{eq:lorentz-transf-def-components}, we can see
    \begin{equation*}
       S \coloneq \eta_{\mu\nu} S^{\mu\nu} \to \eta_{\mu\nu} \tensor{\Lambda}{^\mu_\alpha} \tensor{\Lambda}{^\nu_\beta} S^{\alpha\beta} = \eta_{\alpha\beta} S^{\alpha\beta} = S .
    \end{equation*}
    A faster way to reach this conclusion is to note that, due to the (absence of) index structure of $S$, it is a Lorentz scalar, and so transforms trivially by~\eqref{eq:lorentz-transf-scalar}.

    Concerning the antisymmetric part, a tensor has this property if $A^{\mu\nu} = - A^{\mu\nu}$. Let's verify this is preserved under a Lorentz transformation. A generic two tensor transforms according to eq.~\eqref{eq:lorentz-transf-tensor}, so
    \begin{equation*}
       A^{\mu\nu} \to A'^{\mu\nu} = \tensor{\Lambda}{^\mu_\alpha} \tensor{\Lambda}{^\nu_\beta} A^{\alpha\beta} ,
    \end{equation*}
    and using the antisymmetry of $A^{\mu\nu}$
    \begin{equation*}
        A'^{\nu\mu} = \tensor{\Lambda}{^\nu_\alpha} \tensor{\Lambda}{^\mu_\beta} A^{\alpha\beta} = - \tensor{\Lambda}{^\nu_\alpha} \tensor{\Lambda}{^\mu_\beta} A^{\beta\alpha} = -A'^{\mu\nu} .
    \end{equation*}
    Finally, for the symmetric traceless part,
    \begin{equation*}
        S^{\mu\nu} \to S'^{\mu\nu} = \tensor{\Lambda}{^\mu_\alpha} \tensor{\Lambda}{^\nu_\beta} S^{\alpha\beta} ,
    \end{equation*}
    and using its symmetry
    \begin{equation*}
        S'^{\nu\mu} = \tensor{\Lambda}{^\nu_\alpha} \tensor{\Lambda}{^\mu_\beta} S^{\alpha\beta} = \tensor{\Lambda}{^\nu_\alpha} \tensor{\Lambda}{^\mu_\beta} S^{\beta\alpha} = S'^{\mu\nu} ,
    \end{equation*}
    and the facts that it is traceless and the trace it's a scalar,
    \begin{equation*}
        S' = S = 0,
    \end{equation*}
    which proves $S'^{\mu\nu}$ remains a symetric traceless tensor. 
\end{innerproof}
\end{mdframed}

Then, recalling the definition~\eqref{def:reducible-rep}, the original tensor $T^{\mu\nu}$ lived in a reducible representation, and there are three irreducible subspaces, a one-dimensional subspace, spanned by the trace, a $6$-dimensional one, spanned by the antisymmetric tensors, and a $9$-dimensional one, spanned by traceless symmetric tensors. Using the convention of denoting an irreducible representation with its dimensionality, written in boldface, we have
\begin{equation}
    \vec{4} \otimes \vec{4} = \vec{1} \oplus \vec{6} \oplus \vec{9} .
\end{equation}

A priori, it's not necessary true that those representations are irreducible. One should prove it. Without going into the details, let's just cite that the most general irreducible tensor representations of the Lorentz group are found starting from a generic tensor with an arbitrary number of indices, removing first all traces, and then symmetrizing or antisymmetrising over all pairs of indices.


%%%%%%%%%%%%%%%  DECOMPOSITION %%%%%%%%%%%%%%%%%
\color{red}
\subsection{Decomposition of Lorentz tensors under SO(3)}
Even if the vector representation is an irreducible representation of the Lorentz group, from the point of view of $SO(3)$, which is a subgroup, it is reducible. This will be related to spin.




%%%%%%%%%%%%%%%  SPINORIAL REPRESENTATION %%%%%%%%%%%%%%%%%
\subsection{Spinorial Representation}
\dots
\color{black}

%%%%%%%%%%%%%%%  FIELD REPRESENTATION %%%%%%%%%%%%%%%%%
\subsection{Field Representation}\label{sec:field-representation}
So far, we've been dealing with finite-dimensional representations $(\rho,V)$, that acted on finite-dimensional vector spaces $V$ whose elements were spacetime-independent quantities, transforming like
\begin{equation}\label{eq:transformation-field-component}
   \Phi^i \to  \Phi'^i  =  \tensor{\rho(\Lambda)}{^i_j} \Phi^j.
\end{equation}

However, in quantum field theory we're interested in field, that is, functions of spacetime, $\Phi^i(x)$. They will still transform like~\eqref{eq:transformation-field-component}, but the Lorentz transformation will act on the coordinates too, as $x^\mu \to \tensor{\Lambda}{^\mu_\nu}x^\nu$, with $\tensor{\Lambda}{^\mu_\nu}$ in the vector representation of the group (see sec.~\ref{sec:vector-representation}).

Therefore, we have
\begin{equation}\label{eq:lorentz-field-variation}
   \Phi^i(x) \to \Phi'^i (x) = \tensor{\rho(\Lambda)}{^i_j} \Phi^j(\Lambda^{-1}x) \iff \Phi^i(x) \to \Phi'^i(x') = \tensor{\rho(\Lambda)}{^i_j} \Phi^j(x) .
\end{equation}

\begin{remark}
    We have $\Lambda^{-1}$ because we used Wigner's convention for symmetry operators in quantum mechanics. In particular, for a coordinate transformation $R$, $\vec{x} \to R \vec{x}$, we assign a symmetry operator $U_R$ which acts on quantum states as $\ket{\psi} \to U_R \ket{\psi}$ such that $\bra{\vec{x}}U_R\ket{\psi} = \braket{R^{-1}\vec{x}}{\psi}$, or equivalently on functions as $\psi(\vec{x})\to U_R \psi(\vec{x})=\psi(R^{-1}\vec{x})$.
\end{remark}

Now, since both $\Phi_i$ and the coordinates $x^\mu$ transforms under (two different) Lorentz representations, we can study two types of infinitesimal variations of the field $\Phi^a(x)$ under a transformation:
\begin{itemize}
    \item \emph{Total variation}: represents how $\Phi^i$ change under an infinitesimal Lorentz transformation, at a fixed point $P$,
    \begin{equation}\label{eq:lorentz-total-variation}
        \delta \Phi^i = \Phi'^i(x') - \Phi^i(x) = - \frac{i}{2} \omega_{\mu\nu} \tensor{(S^{\mu\nu})}{^i_j} \Phi^j(x) ,
    \end{equation}
    with $\tensor{(S^{\mu\nu})}{^i_j}$ a finite-dimensional matrix representation of the generators $M^{\mu\nu}$ of the Lorentz group. The letter $S$ stands for \emph{spin}, as will be clear in the following. Here, both $x$ and $x'$ denote the same point $P$ in the different reference frames. Therefore, we're studying how the finite numbers of degrees of freedoms change when we change the label of $P$ from $x$ to $x'$, with $P$ fixed. As usual, the Lorentz indices $\mu\nu$ denote the particular representations, while $ij$ are the representation indices, depending on the vector space $S^{\mu\nu}$ act upon. This transformation is the one we've studied so far, represented by eq.~\eqref{eq:action-lorentz-transf}. It depends on which representation $\Phi^i$ transforms upon, and it is one of the matrices we've studied in the previous sections.
    
    \item \emph{Variation in form}: this is an infinitesimal variation at fixed coordinate $x$, i.e.,
    \begin{equation}\label{eq:lorentz-variation-in-form}
        \delta_0 \Phi^i = \Phi'^i(x) - \Phi^i(x) = \Phi'^i (x' - \delta x) - \Phi^i(x) = \delta \Phi^i - \delta x_\mu \de^\mu \Phi^i ,
    \end{equation}
    where, in the original frame, $x$ denotes a point $P$, while, after the transformation on the coordinates, in the new frame, $x$ denotes a different point $P'$.
\end{itemize}

Basically, $\delta \Phi^i$ always provide finite-dimensional representations of the generators, depending on the degrees of freedom of $\Phi^i$, though of as an object transforming as~\eqref{eq:transformation-field-component}. Indeed, the point $P$ is kept fixed, and the base space is made of $\Phi^i(P)$, and so its dimension is provided by the range of the index $i$, which depends on the representation $\Phi^i$ transforms upon. Conversely, $\delta_0 \Phi^i$ compares the fields at two different space-time points $P$ and $P'$, and so, it compares different degrees of freedom. In particular, the base space is now the set $\Phi^i(P)$, with $P$ varying now over the whole space-time. In other words, the base space is a \emph{space of functions}, and so it's an infinite-dimensional base-space. We then obtain an infinite-dimensional representation of the generators.

At the end, taking into account the variation in form of the field, under a Lorentz transforms, the latter transforms as
\begin{equation}\label{eq:total-lorentz-variation}
    \Phi'^i(x) = \Phi^i - \frac{i}{2} \omega_{\mu\nu} \tensor{\left(S^{\mu\nu} + L^{\mu\nu}\right)}{^i_j} \Phi^j (x)
\end{equation}
where ${S}^{\mu\nu}$ is defined by eq.~\eqref{eq:lorentz-total-variation}, while ${L}^{\mu\nu}$ is defined by
\begin{equation}
    {{L}}^{\mu\nu} \coloneq i (x^\mu \de^\nu - x^\nu \de^\mu), \quad  \tensor{(L^{\mu\nu})}{^i_j} \coloneq L^{\mu\nu} \delta^i_j.
\end{equation}

\begin{mdframed}
\begin{innerproof}
    Recalling that the infinitesimal Lorentz transformation on the coordinates is
    \begin{equation*}
        x^\mu \to \tensor{\Lambda}{^\mu_\nu}x^\nu \simeq \tensor{\omega}{^\mu_\nu}x^\nu \implies \delta x_\mu = \omega_{\mu\nu}x^\nu ,
    \end{equation*}
    we can write
    \begin{equation*}
        -\delta x_\mu \de^\mu \Phi^i = - \omega_{\mu\nu} x^\nu \de^\mu \Phi^i = -\frac{i}{2} \omega_{\mu\nu} \left[ i (x^\mu \de^\nu - x^\nu \de^\mu) \right] \Phi^i ,
    \end{equation*}
    where we added a piece in the parenthesis to make it antisymmetric with respect to $\mu\nu$. Then, inserting it in eq.~\eqref{eq:lorentz-variation-in-form} and using the definition~\eqref{eq:lorentz-total-variation}, we obtain
    \begin{align*}
        \delta_0 \Phi^i = \delta \Phi^i - \delta x_\mu \de^\mu \Phi^i &= -\frac{i}{2} \omega_{\mu\nu} \tensor{(S^{\mu\nu})}{^i_j} \Phi^j(x) -\frac{i}{2} \omega_{\mu\nu} \left[ i (x^\mu \de^\nu - x^\nu \de^\mu) \right] \Phi^i (x).
    \end{align*}
    Therefore, defining
    \begin{equation*}
        {{L}}^{\mu\nu} \coloneq i (x^\mu \de^\nu - x^\nu \de^\mu) ,
    \end{equation*}
    we can write
    \begin{equation*}
        \delta_0 \Phi^i (x) = -\frac{i}{2} \omega_{\mu\nu} \left[ \tensor{(S^{\mu\nu})}{^i_j} + L^{\mu\nu} \delta^i_j \right] \Phi^j (x).
    \end{equation*}
    Finally, defining the diagonal operator 
    \begin{equation*}
        \tensor{(L^{\mu\nu})}{^i_j} \coloneq L^{\mu\nu} \delta^i_j,
    \end{equation*}
    and using the first part of eq.~\eqref{eq:lorentz-variation-in-form}, we write
    \begin{equation*}
        \Phi'^i(x) = \Phi^i(x) + \delta_0 \Phi^i (x) = \Phi^i - \frac{i}{2} \omega_{\mu\nu} \tensor{\left(S^{\mu\nu} + L^{\mu\nu}\right)}{^i_j} \Phi^j (x),
    \end{equation*}
    which concludes the proof.
\end{innerproof}
\end{mdframed}
